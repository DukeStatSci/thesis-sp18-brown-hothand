% This is the Duke University Statistical Science LaTeX thesis template.
% It has been adapted from the Reed College LaTeX thesis template. The
% adaptation was done by Mine Cetinkaya-Rundel (MCR). Some of the comments
% that are specific to Reed College have been removed.
%
% Most of the work on the original Reed College document class and template
% was done by Sam Noble (SN). Later comments etc. by Ben Salzberg (BTS).
% Additional restructuring and APA support by Jess Youngberg (JY).
%
% See https://www.reed.edu/cis/help/latex/ for help. There are a
% great bunch of help pages there, with notes on
% getting started, bibtex, etc. Go there and read it if you're not
% already familiar with LaTeX.
%
% Any line that starts with a percent symbol is a comment.
% They won't show up in the document, and are useful for notes
% to yourself and explaining commands.
% Commenting also removes a line from the document;
% very handy for troubleshooting problems. -BTS

%%
%% Preamble
%%
% \documentclass{<something>} must begin each LaTeX document
\documentclass[12pt,twoside]{dukestatscithesis}
% Packages are extensions to the basic LaTeX functions. Whatever you
% want to typeset, there is probably a package out there for it.
% Chemistry (chemtex), screenplays, you name it.
% Check out CTAN to see: http://www.ctan.org/
%%
\usepackage{graphicx,latexsym}
\usepackage{amsmath}
\usepackage{amssymb,amsthm}
\usepackage{longtable,booktabs,setspace}
\usepackage{chemarr} %% Useful for one reaction arrow, useless if you're not a chem major
\usepackage[hyphens]{url}
% Added by CII
\usepackage{hyperref}
\usepackage{lmodern}
\usepackage{float}
\floatplacement{figure}{H}
% End of CII addition
\usepackage{rotating}

% Next line commented out by CII
%%% \usepackage{natbib}
% Comment out the natbib line above and uncomment the following two lines to use the new
% biblatex-chicago style, for Chicago A. Also make some changes at the end where the
% bibliography is included.
%\usepackage{biblatex-chicago}
%\bibliography{thesis}


% Added by CII (Thanks, Hadley!)
% Use ref for internal links
\renewcommand{\hyperref}[2][???]{\autoref{#1}}
\def\chapterautorefname{Chapter}
\def\sectionautorefname{Section}
\def\subsectionautorefname{Subsection}
% End of CII addition

% Added by CII
\usepackage{caption}
\captionsetup{width=5in}
% End of CII addition

% \usepackage{times} % other fonts are available like times, bookman, charter, palatino


% To pass between YAML and LaTeX the dollar signs are added by CII
\title{My Final College Paper}
\author{Nathaniel Brown}
% The month and year that you submit your FINAL draft TO THE LIBRARY (May or December)
\date{May 20xx}
\advisor{Advisor F. Name}
\institution{Duke University}
\degree{Bachelor of Science in Statistical Science}
\committeememberone{Committeemember O. Name}
\committeemembertwo{Committeemember T. Name}
\dus{Dus X. Name}
%If you have two advisors for some reason, you can use the following
% Uncommented out by CII
% End of CII addition

%%% Remember to use the correct department!
\department{Department of Statistical Science}

% Added by CII
%%% Copied from knitr
%% maxwidth is the original width if it's less than linewidth
%% otherwise use linewidth (to make sure the graphics do not exceed the margin)
\makeatletter
\def\maxwidth{ %
  \ifdim\Gin@nat@width>\linewidth
    \linewidth
  \else
    \Gin@nat@width
  \fi
}
\makeatother

\renewcommand{\contentsname}{Table of Contents}
% End of CII addition

\setlength{\parskip}{0pt}

% Added by CII

\providecommand{\tightlist}{%
  \setlength{\itemsep}{0pt}\setlength{\parskip}{0pt}}

\Acknowledgements{
I want to thank a few people.
}

\Dedication{
You can have a dedication here if you wish.
}

\Preface{
This is an example of a thesis setup to use the reed thesis document
class (for LaTeX) and the R bookdown package, in general.
}

\Abstract{
\chapter{Abstract}\label{abstract}

The proposed study is an investigation of Bayesian statistical models
and analyses for problems arising in shooting a basketball. Goals will
be to explore, develop and apply Bayesian models to existing and new
data on shooting outcomes, to understand and evaluate questions of
inherent random variation, changes over time in shooting performance,
and issues related to the ``Hot Hand'' concept in sports.

\par

(talk about models you fit)

\par

(talk about results)
}

% End of CII addition
%%
%% End Preamble
%%
%

\usepackage{amsthm}
\newtheorem{theorem}{Theorem}[chapter]
\newtheorem{lemma}{Lemma}[chapter]
\theoremstyle{definition}
\newtheorem{definition}{Definition}[chapter]
\newtheorem{corollary}{Corollary}[chapter]
\newtheorem{proposition}{Proposition}[chapter]
\theoremstyle{definition}
\newtheorem{example}{Example}[chapter]
\theoremstyle{definition}
\newtheorem{exercise}{Exercise}[chapter]
\theoremstyle{remark}
\newtheorem*{remark}{Remark}
\newtheorem*{solution}{Solution}
\begin{document}

% Everything below added by CII
  \maketitle

\frontmatter % this stuff will be roman-numbered
\pagestyle{empty} % this removes page numbers from the frontmatter
  \begin{acknowledgements}
    I want to thank a few people.
  \end{acknowledgements}
  \begin{preface}
    This is an example of a thesis setup to use the reed thesis document
    class (for LaTeX) and the R bookdown package, in general.
  \end{preface}
  \hypersetup{linkcolor=black}
  \setcounter{tocdepth}{2}
  \tableofcontents

  \listoftables

  \listoffigures
  \begin{abstract}
    \chapter{Abstract}\label{abstract}
    
    The proposed study is an investigation of Bayesian statistical models
    and analyses for problems arising in shooting a basketball. Goals will
    be to explore, develop and apply Bayesian models to existing and new
    data on shooting outcomes, to understand and evaluate questions of
    inherent random variation, changes over time in shooting performance,
    and issues related to the ``Hot Hand'' concept in sports.
    
    \par
    
    (talk about models you fit)
    
    \par
    
    (talk about results)
  \end{abstract}
  \begin{dedication}
    You can have a dedication here if you wish.
  \end{dedication}
\mainmatter % here the regular arabic numbering starts
\pagestyle{fancyplain} % turns page numbering back on

\chapter*{Introduction}\label{introduction}
\addcontentsline{toc}{chapter}{Introduction}

Welcome to the \emph{R Markdown} thesis template. This template is based
on (and in many places copied directly from) the Reed College LaTeX
template, but hopefully it will provide a nicer interface for those that
have never used TeX or LaTeX before. Using \emph{R Markdown} will also
allow you to easily keep track of your analyses in \textbf{R} chunks of
code, with the resulting plots and output included as well. The hope is
this \emph{R Markdown} template gets you in the habit of doing
reproducible research, which benefits you long-term as a researcher, but
also will greatly help anyone that is trying to reproduce or build onto
your results down the road.

Hopefully, you won't have much of a learning period to go through and
you will reap the benefits of a nicely formatted thesis. The use of
LaTeX in combination with \emph{Markdown} is more consistent than the
output of a word processor, much less prone to corruption or crashing,
and the resulting file is smaller than a Word file. While you may have
never had problems using Word in the past, your thesis is likely going
to be about twice as large and complex as anything you've written
before, taxing Word's capabilities. After working with \emph{Markdown}
and \textbf{R} together for a few weeks, we are confident this will be
your reporting style of choice going forward.

\textbf{Why use it?}

\emph{R Markdown} creates a simple and straightforward way to interface
with the beauty of LaTeX. Packages have been written in \textbf{R} to
work directly with LaTeX to produce nicely formatting tables and
paragraphs. In addition to creating a user friendly interface to LaTeX,
\emph{R Markdown} also allows you to read in your data, to analyze it
and to visualize it using \textbf{R} functions, and also to provide the
documentation and commentary on the results of your project. Further, it
allows for \textbf{R} results to be passed inline to the commentary of
your results. You'll see more on this later.

Having your code and commentary all together in one place has a plethora
of benefits!

\textbf{Who should use it?}

Anyone who needs to use data analysis, math, tables, a lot of figures,
complex cross-references, or who just cares about the final appearance
of their document should use \emph{R Markdown}. Of particular use should
be anyone in the sciences, but the user-friendly nature of
\emph{Markdown} and its ability to keep track of and easily include
figures, automatically generate a table of contents, index, references,
table of figures, etc. should make it of great benefit to nearly anyone
writing a thesis project.

\chapter{Abstract}\label{abstract}

The proposed study is an investigation of Bayesian statistical models
and analyses for problems arising in shooting a basketball. Goals will
be to explore, develop and apply Bayesian models to existing and new
data on shooting outcomes, to understand and evaluate questions of
inherent random variation, changes over time in shooting performance,
and issues related to the ``Hot Hand'' concept in sports.

\par

(talk about models you fit)

\par

(talk about results)

\chapter{Literature Review}\label{litreview}

Gilovich, Vallone, \& Tversky (1985)

A. survey of basketball fans
\begin{enumerate}
\def\labelenumi{\arabic{enumi}.}
\tightlist
\item
  qualitative questions, opinions
\item
  hypothetical preditions (FT\% on \#2 given \#1 was a hit)
\item
  streak detection
  \begin{enumerate}
  \def\labelenumii{\alph{enumii}.}
  \tightlist
  \item
    asked to classify sequences as ``chance'', ``streak'', or
    ``alternate''
  \item
    alternation probability = 0.5 was considered streaky by 62\% of
    subjects
  \item
    people not only perceive random sequences as positively correlated,
    they also perceive negatively correlated sequences as random
  \end{enumerate}
\end{enumerate}
B. FG analysis from NBA games
\begin{enumerate}
\def\labelenumi{\arabic{enumi}.}
\tightlist
\item
  autocorrelation between conditional probs P(H\textbar{}M1), P(H),
  P(H\textbar{}H1), etc.
\item
  paired t-test between the diff conditional probs (like
  P(H\textbar{}M1) and P(H\textbar{}H1))
\item
  Wald-Wolfowitz run test
\item
  Stationarity:
  \begin{enumerate}
  \def\labelenumii{\alph{enumii}.}
  \tightlist
  \item
    Chi-Sq test to look at if disjoint bins of shots followed expected
    distribution of ``high'', ``med'', and ``low'' performance.
  \end{enumerate}
\item
  Hot and Cold Night variability
  \begin{enumerate}
  \def\labelenumii{\alph{enumii}.}
  \tightlist
  \item
    look at flucutation between std erros of games and std error of
    season
  \item
    Lexis ratio (SE obs / SE expc) should be significantly
    \textgreater{} 1
  \end{enumerate}
\end{enumerate}
C. FT analysis from NBA games
\begin{enumerate}
\def\labelenumi{\arabic{enumi}.}
\item
\begin{verbatim}
autocorrelation again
\end{verbatim}
\end{enumerate}
\begin{enumerate}
\def\labelenumi{\Alph{enumi}.}
\setcounter{enumi}{3}
\tightlist
\item
  controlled experiment with Cornell basketball player.
\end{enumerate}
\begin{enumerate}
\def\labelenumi{\arabic{enumi}.}
\item
\begin{verbatim}
 process:
\end{verbatim}
  \begin{enumerate}
  \def\labelenumii{\alph{enumii}.}
  \item
    For each player we determined a distance from which his or her
    shooting percentage was roughly 50\%. At this distance we then drew
    two 15-ft arcs on the floor from which each player took all of his
    or her shots. The centers of the arcs were located 60 degrees out
    from the left and right sides of the basket. When shooting baskets,
    the players were required to move along the arc between shots so
    that consecutive shots were never taken from exactly the same spot.
    Each player was to take 100 shots, 50 from each arc.
  \item
    We tested players' ability to predict hits and misses by having them
    bet on the outcome of each upcoming shot. Before every shot, each
    player chose whether to bet high in which case he or she would win 5
    cents for a hit and lose 4 cents for a miss; or bet low, in which
    case he or she would win 2 cents for a hit and lose 1 cent for a
    miss. The players were advised to bet high when they felt confident
    in their shooting ability, and to bet low when they did not. We also
    obtained betting data fro another player who observed the shooter.
  \end{enumerate}
\item
  data analysis was the same tests as NBA FG
\item
  reports prediction correlations with result, other predictions, and
  prev result, etc.
\end{enumerate}
E. Conclusion
\begin{enumerate}
\def\labelenumi{\arabic{enumi}.}
\tightlist
\item
  there is no hot hand.
\end{enumerate}
Albert \& Williamson (1999)

A. Introduction
\begin{enumerate}
\def\labelenumi{\arabic{enumi}.}
\tightlist
\item
  discussion of previous research.
  \begin{enumerate}
  \def\labelenumii{\alph{enumii}.}
  \tightlist
  \item
    the 1985 Gilovich tests lacked statistical power
  \end{enumerate}
\item
  streakiness is the presence of nonstationarity or autocorrelation in a
  binary sequence:
  \begin{enumerate}
  \def\labelenumii{\alph{enumii}.}
  \tightlist
  \item
    nonstationarity (nonconstant prob between trials)
  \item
    can be modeled with overdispersion model
  \item
    autocorrelation (sequential dependency)
  \end{enumerate}
\end{enumerate}
B. Simulating models and data
\begin{enumerate}
\def\labelenumi{\arabic{enumi}.}
\tightlist
\item
  Using ``nearly'' sufficient statistics to approximate posterior
  density of interest, condidional on the statistic falline in a small
  interval about the observed value.
\item
  simulate using regular gibbs algorithm
\end{enumerate}
C. Fitting overdispersion model (on data where one suspects
nonstationarity)
\begin{enumerate}
\def\labelenumi{\arabic{enumi}.}
\tightlist
\item
  Mike Schmidt batting data
  \begin{enumerate}
  \def\labelenumii{\alph{enumii}.}
  \tightlist
  \item
    14 baseball seasons, split each into 13 2-week sets, count num HR
    per set
  \item
    observed that var(y) \textgreater{} mean(y)
  \item
    fit overdispersion mixture model on poisson y.
  \end{enumerate}
\end{enumerate}
D. Another example (from a baseballer with a reputation of streakiness)
\begin{enumerate}
\def\labelenumi{\arabic{enumi}.}
\tightlist
\item
  Javy Lopez 1998 hitting data
  \begin{enumerate}
  \def\labelenumii{\alph{enumii}.}
  \tightlist
  \item
    num hits and num at-bats for each of 132 games of the season
  \end{enumerate}
\item
  how do we measure streakiness/non-homogeneity?
  \begin{enumerate}
  \def\labelenumii{\alph{enumii}.}
  \tightlist
  \item
    sufficient stats included:
  \item
    T1 = difference between max and min moving average (10-game bins)
  \end{enumerate}
  \begin{enumerate}
  \def\labelenumii{\roman{enumii}.}
  \setcounter{enumii}{1}
  \item
    T2 = sum of the abs difference of moving avgs and season avg
  \item
    T3 = number of ``runs'' (consecutive 1s and 0s) in sequence
  \item
    T4 = length of the longest run in seq
  \item
    T5 = number of runs exceeding a specified length
  \item
    T6 = \(log[(n_{00}n_{11})/(n_{01}n_{10})]\) \(n_{ij}\) = ijth index
    of matrix classifying game i and j as hot or cold. (j=i+1)

    T6 \textgreater{} 0 if strong autocorrelation
  \item
    using logistic function where p = hit\% and xk = batting avg for
    last k games; logit(p) = B0 + B1*xk T7(k) = B1 \textgreater{} 0
  \item
    T8 = sd(p1, \ldots{}, pN) (given N bins of batting data)
  \end{enumerate}
\item
  Markov switching model
  \begin{enumerate}
  \def\labelenumii{\alph{enumii}.}
  \tightlist
  \item
    two hitting states \(p_{hot}\) and \(p_{cold}\)
  \item
    compared to ``coin toss'' model with constant prbability
  \item
    simulating results of games showed that there is not evidence for
    hot hand
  \end{enumerate}
\end{enumerate}
E. Free throw shooting
\begin{enumerate}
\def\labelenumi{\arabic{enumi}.}
\tightlist
\item
  data
  \begin{enumerate}
  \def\labelenumii{\alph{enumii}.}
  \tightlist
  \item
    College bball player shot 100 shots for 20 days. 20 x 100 matrix
  \end{enumerate}
\item
  model
  \begin{enumerate}
  \def\labelenumii{\alph{enumii}.}
  \tightlist
  \item
    dispersion instead of Markov switching
  \end{enumerate}
\item
  analysis:
  \begin{enumerate}
  \def\labelenumii{\alph{enumii}.}
  \tightlist
  \item
    group 100 shots for each day into 20 bins of 5. (i \textless{}= 20,
    j \textless{}= 20)
  \item
    record proportions \(hat{y_{1}}...hat{y_20}\)
  \end{enumerate}
\item
  results
  \begin{enumerate}
  \def\labelenumii{\alph{enumii}.}
  \tightlist
  \item
    dispersion decreases as days go on. (practice makes perfect)
  \end{enumerate}
\end{enumerate}
Koehler \& Conley (2003)

A. prior research

B. procedure
\begin{enumerate}
\def\labelenumi{\arabic{enumi}.}
\tightlist
\item
  analyze NBA 3-pt contest footage (4 years, 23 shooters (some rpt
  contestants), 56 total rounds)
\item
  search for seq dependency within shooters across all shots, and per
  set of 25 shots
\end{enumerate}
C. results
\begin{enumerate}
\def\labelenumi{\arabic{enumi}.}
\tightlist
\item
  no evidence of hotness or seq dependencies
\item
  runs analysis from each shooter
\item
  conditional probability analysis
\item
  analysis of how they shot when announcers said a player was getting
  hot
  \begin{enumerate}
  \def\labelenumii{\alph{enumii}.}
  \tightlist
  \item
    the 3 before the announcer jinx were high but not the one
    immediately after.
  \end{enumerate}
\end{enumerate}
D. discussion
\begin{enumerate}
\def\labelenumi{\arabic{enumi}.}
\tightlist
\item
  recent streaks of success are not predictive of future performance;
  use base rate
\item
  irrational hot hand belief may lead to exploiting market
  inefficiencies
\item
  future research might focus more on implications of the false belief
  in the hot hand rather than on pinpointing where hotness exists
\end{enumerate}
Bar-Eli, Avugos, \& Raab (2006)
\begin{enumerate}
\def\labelenumi{\Alph{enumi}.}
\tightlist
\item
  Introduction
\end{enumerate}
\begin{enumerate}
\def\labelenumi{\arabic{enumi}.}
\tightlist
\item
  a review of 20 years of hot hand research.
\item
  in short: 11 support, 13 not support
  \begin{enumerate}
  \def\labelenumii{\alph{enumii}.}
  \tightlist
  \item
    but the supportive ones are typically weaker/have questionable data
    or unrealistic model
  \end{enumerate}
\item
  so far, the scientific support for the hot hand is controversial and
  fairly limited
\item
  but the belief in it is important.
\end{enumerate}
B. Preliminary Hot hand research
\begin{enumerate}
\def\labelenumi{\arabic{enumi}.}
\tightlist
\item
  assumption that prob of success is affected by physical or
  psychological state.
\item
  the scientific norm of randomness may present it as a fallacy but it's
  useful because it affects real life decisions
\item
  after gilovich 1985 paper, more applications appeared
  \begin{enumerate}
  \def\labelenumii{\alph{enumii}.}
  \tightlist
  \item
    other sports: baseball, volleyball, golf, tennis, bowling, darts,
    horseshoes
  \item
    non sports: economics, cognitive science, law, religion
  \end{enumerate}
\end{enumerate}
C. The Origin of the belief
\begin{enumerate}
\def\labelenumi{\arabic{enumi}.}
\tightlist
\item
  memory bias: streaks are more memorable and thus wrongly regarded as
  nonrandom
\item
  general misconception: people often disregard sample size. people
  expect more alternations than what is normal.
  \begin{enumerate}
  \def\labelenumii{\alph{enumii}.}
  \tightlist
  \item
    law of small numbers (fallicious belief that law of large numbers
    applies to small samples)
  \end{enumerate}
\item
  gambler's fallacy (states that a streak of events is likely to end)
  \begin{enumerate}
  \def\labelenumii{\alph{enumii}.}
  \tightlist
  \item
    basically the opposite of the hot hand fallacy
  \item
    but both can be explained by ``representativeness'' and sequence
    recency.
  \item
    they're interpreted inversely because slot machines are seen as
    unintentional while shooting a basketball is seen as intentional.
  \end{enumerate}
\end{enumerate}
D. The validity of the Belief
\begin{enumerate}
\def\labelenumi{\arabic{enumi}.}
\tightlist
\item
  studies that provide evidence against some investigate the link
  between belief and behavior instead of just a simple ``is it real?''
\item
  evidence for
  \begin{enumerate}
  \def\labelenumii{\alph{enumii}.}
  \tightlist
  \item
    Larkey et al 1989:
  \item
    they used shorter chunks within a game (seq of 20 baskets) because
    extracting individual sequences from an actual game is too
    complicated
  \end{enumerate}
  \begin{enumerate}
  \def\labelenumii{\roman{enumii}.}
  \setcounter{enumii}{1}
  \tightlist
  \item
    found out that Vinnie Johnson was indeed streaky
  \item
    but they only used 7 makes within a string of 20 in one game.
  \item
    and turns out that the stats were incorrectly recorded and the 7 in
    a row never even happened!
  \end{enumerate}
  \begin{enumerate}
  \def\labelenumii{\alph{enumii}.}
  \setcounter{enumii}{1}
  \tightlist
  \item
    Waldrop 1995
  \item
    interesting insight about the Simpson's paradox and how shooting
    better after a make may be seen in aggregated data but not
    individually.
  \item
    Gilden and Wilson (1995)
  \item
    asked random ppl to do darts and putts, but with little compensation
  \end{enumerate}
  \begin{enumerate}
  \def\labelenumii{\roman{enumii}.}
  \setcounter{enumii}{1}
  \tightlist
  \item
    therefore streakiness could be lack of motivation
  \item
    also sample size was huge, which means lower p-vals in general
  \item
    Clark 2003
  \item
    PGA stuff. Any clustering of scores thought to be explained by
    hotness was actually explained by course difficulty
  \end{enumerate}
  \begin{enumerate}
  \def\labelenumii{\alph{enumii}.}
  \setcounter{enumii}{4}
  \tightlist
  \item
    Klaasen and Magnus 2001
  \item
    very very very tiny momentum effect in tennis
  \item
    Frame et al 2003
  \item
    basketball, baseball, tennis results are dep. on strategy of
    opposition
  \end{enumerate}
  \begin{enumerate}
  \def\labelenumii{\roman{enumii}.}
  \setcounter{enumii}{1}
  \tightlist
  \item
    there was modest evidence in horsehoe tossing tho!!!
  \end{enumerate}
  \begin{enumerate}
  \def\labelenumii{\alph{enumii}.}
  \setcounter{enumii}{6}
  \tightlist
  \item
    Dorsey-Palmateer and Smith 2004
  \item
    results bowling games exhibited some hotness (maybe just because
    next opponent was sitting and waiting)
  \end{enumerate}
  \begin{enumerate}
  \def\labelenumii{\roman{enumii}.}
  \setcounter{enumii}{1}
  \tightlist
  \item
    results of indiv rolls within games did not exhibit this.
  \end{enumerate}
\end{enumerate}
E. Hot Hand Test Statistics
\begin{enumerate}
\def\labelenumi{\arabic{enumi}.}
\tightlist
\item
  Gilovich 1985 (used simplification of binomial model; has questionable
  power.)
  \begin{enumerate}
  \def\labelenumii{\alph{enumii}.}
  \tightlist
  \item
    prob of hit conditioned on results of previous shot(s)
  \item
    first-order correlation coef
  \item
    Wald-Wolfowitz runs analysis test
  \item
    grouping blocks of 4 shots
  \end{enumerate}
\item
  Wardrop (1999) simulated streaky data to see how binom tests detected
  it
  \begin{enumerate}
  \def\labelenumii{\alph{enumii}.}
  \tightlist
  \item
    it wasn't good; they didn't detect unless the change was extreme,
    partially because of small sample sizes too
  \end{enumerate}
\item
  Miyoshi (2000) suggested 4 variables that affected power of model:
  \begin{enumerate}
  \def\labelenumii{\alph{enumii}.}
  \tightlist
  \item
    frequency of hot periods
  \item
    number of shots in all hot periods for a season
  \item
    number of shots in each hot period
  \item
    magnitude of probability increase during a hot period
  \item
    with these parameters set realistically, Gilovich's power would have
    only been 0.12
  \end{enumerate}
\item
  so no matter how you slice it, it's difficult to model basketball
  events
  \begin{enumerate}
  \def\labelenumii{\alph{enumii}.}
  \tightlist
  \item
    the alternative hypothesis is not that streakines exists, but that
    it exists for everybody in the same way. this is unreasonable.
  \item
    an effective and powerful test for one person may not work for
    another
  \item
    1 person may exhibit non stationarity while another exhibits
    autocorrelation.
  \end{enumerate}
\end{enumerate}
F. Implications of hot hand belief
\begin{enumerate}
\def\labelenumi{\arabic{enumi}.}
\tightlist
\item
  sports gambling
  \begin{enumerate}
  \def\labelenumii{\alph{enumii}.}
  \tightlist
  \item
    the market believes it
  \item
    teams on winning streaks marginally underperform \& vice versa
  \end{enumerate}
\item
  game strategy
  \begin{enumerate}
  \def\labelenumii{\alph{enumii}.}
  \tightlist
  \item
    hot players are more likely to force bad shots
  \item
    hotness should be taken as a piece of information including base
    rate, def strategy, player strengths, other stuff; it should not
    replace your game plan tho.
  \end{enumerate}
\end{enumerate}
\begin{enumerate}
\def\labelenumi{\Alph{enumi}.}
\setcounter{enumi}{6}
\tightlist
\item
  Discussion
\end{enumerate}
\begin{enumerate}
\def\labelenumi{\arabic{enumi}.}
\tightlist
\item
  Task segregation
  \begin{enumerate}
  \def\labelenumii{\alph{enumii}.}
  \tightlist
  \item
    baseball \& basketball are the most popular experiments, but
    repetitive solo events like darts \& golf make more sense;
    horseshoes \& bowling had strongest evidence
  \item
    quality of data was often poor
  \item
    the supportive studies used professional
  \item
    IF hotness exists, then it depends on much more than has been
    controlled for in prev studies
  \item
    participant expertise, task difficulty, atmosphere, motivation, etc.
  \end{enumerate}
\item
  hot hand definitions
  \begin{enumerate}
  \def\labelenumii{\alph{enumii}.}
  \tightlist
  \item
    most common test is the runs test that Gilovich used
  \item
    Hales (1999) defined it in the following terms (none of which were
    found in sports data)
  \item
    success breeds success
  \end{enumerate}
  \begin{enumerate}
  \def\labelenumii{\roman{enumii}.}
  \setcounter{enumii}{1}
  \tightlist
  \item
    streaks are statistically unlikely
  \item
    num. of consecutive successes must exceed those predicted by chance
  \item
    Hales argued that ``being hot does not have to do with the
    fecundity, duration, or even frequency of streaks. It has to do with
    their existence'' (p.~86).
  \item
    When one believes he has a hot hand, he may well be usually right.
  \end{enumerate}
  \begin{enumerate}
  \def\labelenumii{\alph{enumii}.}
  \setcounter{enumii}{4}
  \tightlist
  \item
    some say ``a player has a hot hand when he is playing better than
    average''
  \end{enumerate}
\item
  the cold hand
  \begin{enumerate}
  \def\labelenumii{\alph{enumii}.}
  \tightlist
  \item
    hot hand is based on notion that success motivates ppl and raises
    confidence
  \item
    Berry and Wood (2004) found that icing the kicker in NFL is
    effective strategy
  \end{enumerate}
\item
  Evaluation procedure
  \begin{enumerate}
  \def\labelenumii{\alph{enumii}.}
  \tightlist
  \item
    eval. of indiv. studies is essential 4 future research (wide range
    of outcomes)
  \item
    but also consider big picture
  \item
    good idea to combine and reanalyze data from separate studies that
    used same sport and got diff conclusions.
  \end{enumerate}
\end{enumerate}
H. Concluding Remarks
\begin{enumerate}
\def\labelenumi{\arabic{enumi}.}
\tightlist
\item
  this paper looked at studies that collected real data and simmed data
\item
  question remains unanswered
\item
  Gilovich somewhat strongly showed evidence against seq dep. but the
  nonstate test wasn't powerful enough according to simulation studies.
\item
  stepping further than the ``is there a hot hand?'' question
  \begin{enumerate}
  \def\labelenumii{\alph{enumii}.}
  \tightlist
  \item
    investigating the norms used by people to justify belief in it
  \item
    look at situational factors that enable us to judge the value of the
    belief (strategic advantage vs detrimental fallacy)
  \item
    investigate how important it is to real decisions (gambling,
    coaching, etc.)
  \item
    if hot hand doesn't exist, then training and preparation and sports
    psych methods should reflect this (but they don't, of course)
  \end{enumerate}
\end{enumerate}
Ryan Wetzels (2016)

A. Introduction
\begin{enumerate}
\def\labelenumi{\arabic{enumi}.}
\tightlist
\item
  (past studies and stuff you've already read)
\item
  Albert (1993) and Markov logic
  \begin{enumerate}
  \def\labelenumii{\alph{enumii}.}
  \tightlist
  \item
    difference: albert used constant ?? = p(switch) = 0.1, but makes ??
    a free param
  \item
    diff: albert may switch states btwn games, wetzels may switch @
    arbitrary
  \item
    similarity: first order markov chain (pi only depends on pi-1)
  \end{enumerate}
\end{enumerate}
B. A two-state Bernoulli hidden markov model
\begin{enumerate}
\def\labelenumi{\arabic{enumi}.}
\tightlist
\item
  consider first order hidden markov model (HMM)
  \begin{enumerate}
  \def\labelenumii{\alph{enumii}.}
  \tightlist
  \item
    binary state (S) at each timeunit (t). St = 1 ??? hot; St = 0 ???
    cold
  \item
    if switching prob \textless{} 0.5 ??? ``sticky'' states (exhibits
    streakiness) (free param!)
  \end{enumerate}
  \begin{enumerate}
  \def\labelenumii{\roman{enumii}.}
  \setcounter{enumii}{2}
  \tightlist
  \item
    the data Yt comes from Bern dist with different hot and cold probs
  \end{enumerate}
\end{enumerate}
C. simulation study to assess performance of Bayesian test
\begin{enumerate}
\def\labelenumi{\arabic{enumi}.}
\tightlist
\item
  evidence of HMM in favor of CPM will be assed by bayes factor
  \begin{enumerate}
  \def\labelenumii{\alph{enumii}.}
  \tightlist
  \item
    BF = ratio of marginal likeyhoods = p(Y(T) = y(T) \textbar{} HMM) /
    p(Y(T) = y(T) \textbar{} CPM)
  \end{enumerate}
\end{enumerate}
D. application: free throw shooting of Shaq n Kobe
\begin{enumerate}
\def\labelenumi{\arabic{enumi}.}
\tightlist
\item
  Kobe's are closer to CPM according to Bayes Factor test
\item
  Shaq's are closer to HMM according to Bayes Factors
\end{enumerate}
E. application: perceptual identification test from test of visual
discernment test
\begin{enumerate}
\def\labelenumi{\arabic{enumi}.}
\tightlist
\item
  BF: 42\% showed evidence for HMM; 22\% for CPM (rest weren't strong
  either way)
\item
  WW-runs test: 47\% were streaky (rest EITHER not strong or strong for
  CPM)
  \begin{enumerate}
  \def\labelenumii{\alph{enumii}.}
  \tightlist
  \item
    much overlap in the accept/reject regions of both tests.
  \end{enumerate}
\end{enumerate}
F. Conclusion and ramifications
\begin{enumerate}
\def\labelenumi{\arabic{enumi}.}
\item
  restrictions on BF test:
  \begin{enumerate}
  \def\labelenumii{\alph{enumii}.}
  \tightlist
  \item
    p(switch) = alpha \textless{} 0.5 to ensure stickiness
  \end{enumerate}
\item
\begin{verbatim}
comments on BF test
\end{verbatim}
  \begin{enumerate}
  \def\labelenumii{\alph{enumii}.}
  \tightlist
  \item
    HMM was not good at simulating data.concludes that ``streakiness''
    hurts CPM more than it helps HMM.
  \item
    BFs are a measure of relative support, so one model being bad does
    not mean the other one is good, even though it may look that way by
    BF
  \end{enumerate}
\item
\begin{verbatim}
future improvements
\end{verbatim}
  \begin{enumerate}
  \def\labelenumii{\alph{enumii}.}
  \tightlist
  \item
    subject-specific priors can create a more power, instead of indep
    unif dists.
  \item
    continuous variables instead of binary variables?
  \item
    allowing switching prob to depend on the state may improve HMM
  \item
    hierarchichal structure that accounts for season effects and indiv
    effects
  \end{enumerate}
\end{enumerate}
Miller \& Sanjurjo (2016)
\begin{enumerate}
\def\labelenumi{\Alph{enumi}.}
\tightlist
\item
  Intro
\end{enumerate}
\begin{enumerate}
\def\labelenumi{\arabic{enumi}.}
\item
\begin{verbatim}
Inherent subtle but substantial selection bias when measuring conditional dependence of current outcomes on past outcomes in sequential data.
\end{verbatim}
\item
\begin{verbatim}
We prove that for any finite sequence of binary data, in which each outcome of "success" or "failure" is determined by an i.i.d. random variable, the proportion of 1s among the outcomes that immediately follow a streak of consecutive 1s is expected to be strictly less than the underlying (conditional) probability of success.
\end{verbatim}
\item
\begin{verbatim}
P(H) > P(H | H) > P(H | H, H) > P(H | H, H, ... Hk) > ...
\end{verbatim}
\item
\begin{verbatim}
bias decreases as sequence (experimental trials, num coinflips) gets longer; bias increases as streak (condition, num heads) size increased
\end{verbatim}
\item
\begin{verbatim}
example table with n=3, k=1
\end{verbatim}
  \begin{enumerate}
  \def\labelenumii{\alph{enumii}.}
  \tightlist
  \item
    for HHH and HHT, there are two flips after the streak prereq
  \item
    for THT, HTT, HTH, THH, there is one flip after the streak
  \item
    for TTH, TTT, there are zero
  \item
    each seq has an equal prob of happening, but in group a. you record
    2 events, b. is 1, and c. is 0. Therefore, the group a. flips are
    individually weighted the least.
  \item
    This means that individual flips among a small group of heads
    inherently get a lower weight. --\textgreater{} heads get a lower
    prob.
  \end{enumerate}
\item
\begin{verbatim}
how to unbias
\end{verbatim}
  \begin{enumerate}
  \def\labelenumii{\alph{enumii}.}
  \tightlist
  \item
    flip indefinitely until you get m trials that follow a streak,
    instead of flipping exactly n times (negbinom sampling)
  \item
    eliminate overlap by dividing your n trials into bins (that may or
    may not break up some streaks) (explanation was confusing)
  \end{enumerate}
\item
\begin{verbatim}
the bias implies that streaks within finite sequences are expected to end more often than continue (relative to the underlying probability)
\end{verbatim}
  \begin{enumerate}
  \def\labelenumii{\alph{enumii}.}
  \tightlist
  \item
    which can lead both the gambler to think that an i.i.d process has a
    tendency towards reversal,
  \item
    and the hot hand researcher to think that a process is i.i.d. when
    it actually has a tendency towards momentum.
  \end{enumerate}
\item
\begin{verbatim}
you should write a function to simulate this using binomial trials!
\end{verbatim}
\end{enumerate}
\begin{enumerate}
\def\labelenumi{\Alph{enumi}.}
\setcounter{enumi}{1}
\tightlist
\item
  Bias (theoretical results)
\end{enumerate}
\begin{enumerate}
\def\labelenumi{\arabic{enumi}.}
\item
\begin{verbatim}
proof
\end{verbatim}
  \begin{enumerate}
  \def\labelenumii{\alph{enumii}.}
  \item
    show that: if X is binary seq of iid randvars s/t p = p(x\_i = 1) =
    p(x\_i = 1\textbar{} k-length streak), then this procedure yields a
    biased estimate of the conditional probability
  \item
\begin{verbatim}
the probability of seeing a streak is < p, ==> E[p(x=1|streak)] < p
\end{verbatim}
  \end{enumerate}
\item
\begin{verbatim}
relationship to know results (like SWOR)
\end{verbatim}
  \begin{enumerate}
  \def\labelenumii{\alph{enumii}.}
  \item
    any binary seq of n trials will have n1 successes and n0 := n-n1
    failures
  \item
    prior odds of success = n1/n0
  \item
    for trials following streak of k 1s ( denoted in paper as I\_1k(x)
    ), odds dec.
  \item
    prior odds have 2 updating factors \& they both cause drop in odds
    when t \textless{} n
  \item
\begin{verbatim}
first factor: (n1-k)/n1 < 1
\end{verbatim}
  \end{enumerate}
  \begin{enumerate}
  \def\labelenumii{\roman{enumii}.}
  \setcounter{enumii}{1}
  \item
    reflects constraint of the finite number of available trials to
    select from
  \item
    info gained upon learning k of the n1 successes are no longer
    available, which leads to a SWOR effect on the prior odds
  \item
    bias increases as k (streak length) increases
  \item
\begin{verbatim}
second factor: too complicated to type out. some ratio of expectations.
\end{verbatim}
  \item
    reflects constraint of arrangement of successes in the seq
  \item
    info gained upon learning that k 1s are consec. and precede t.
  \item
    if k-streak occurs, x(t+1) WILL be in I\_1k, and
    x(t+2),\ldots{},x(t+k) MIGHT be. else, x(t+1), \ldots{}, x(t+k)
    CANNNOT be in I\_1k.
  \end{enumerate}
  \begin{enumerate}
  \def\labelenumii{\alph{enumii}.}
  \setcounter{enumii}{4}
  \tightlist
  \item
    SWOR formula
  \item
    bias is determined not by size of n1, but by number of (overlapping)
    instances of k consecutive 1s in the first n???1 trials, which
    depends on both n1 and their arr.
  \item
    overlapping words paradox?
  \end{enumerate}
\item
\begin{verbatim}
Quantifying the Bias
\end{verbatim}
  \begin{enumerate}
  \def\labelenumii{\alph{enumii}.}
  \tightlist
  \item
    enumerating all possible seqs for given n and n1 is computationally
    expensive
  \item
    cool magnitude of bias plots
  \item
    visualizing P1k - P0k
  \end{enumerate}
\end{enumerate}
\begin{enumerate}
\def\labelenumi{\Alph{enumi}.}
\setcounter{enumi}{2}
\tightlist
\item
  Hot Hand Application
\end{enumerate}
\begin{enumerate}
\def\labelenumi{\arabic{enumi}.}
\item
\begin{verbatim}
debiasing GVT data/tests
\end{verbatim}
  \begin{enumerate}
  \def\labelenumii{\alph{enumii}.}
  \item
    t-test
  \item
\begin{verbatim}
shift the estimated difference Dk := p1k - p0k
\end{verbatim}
  \end{enumerate}
  \begin{enumerate}
  \def\labelenumii{\roman{enumii}.}
  \setcounter{enumii}{1}
  \tightlist
  \item
    ``the bias adjustment is made by subtracting the expected difference
    (a player's overall percentage) from each player's observed
    difference.
  \item
    ``results in 19 of the 25 players exhibiting hot hand shooting''
    p\textless{}0.01
  \end{enumerate}
  \begin{enumerate}
  \def\labelenumii{\alph{enumii}.}
  \setcounter{enumii}{1}
  \item
    permutations test
  \item
\begin{verbatim}
permutation test is invulnerable to the bias
\end{verbatim}
  \end{enumerate}
  \begin{enumerate}
  \def\labelenumii{\roman{enumii}.}
  \setcounter{enumii}{1}
  \tightlist
  \item
    under the null of constant prob, each perm. of 1s and 0s is eq.
    likely
  \item
    process: . record n1 hits and n0 misses . calculate Dk for each
    possible arrangement of the sequence . the distribution of Dks
    should be left skewed . use dist to find how ``significant''ly off
    obs Dk is. . agreeable with t-test thing
  \end{enumerate}
\item
\begin{verbatim}
debiasing other studies
\end{verbatim}
  \begin{enumerate}
  \def\labelenumii{\alph{enumii}.}
  \item
    these studies are:
  \item
\begin{verbatim}
Jagacinski et al. (1979) (read this!!!)
\end{verbatim}
  \end{enumerate}
  \begin{enumerate}
  \def\labelenumii{\roman{enumii}.}
  \setcounter{enumii}{1}
  \tightlist
  \item
    Koehler and Conley (2003), (the three-point shootout one)
  \item
    Avugos et al. (2013a) (no raw data provided ???)
  \item
    and Miller and Sanjurjo (2014). (read this too)
  \end{enumerate}
  \begin{enumerate}
  \def\labelenumii{\alph{enumii}.}
  \setcounter{enumii}{1}
  \item
    2003:
  \item
\begin{verbatim}
only median 49 shots per player. severly underpowered.
\end{verbatim}
  \end{enumerate}
  \begin{enumerate}
  \def\labelenumii{\roman{enumii}.}
  \setcounter{enumii}{1}
  \item
    when debiasing, there is a significant improvement.
  \item
    1979 \& 2014:
  \item
\begin{verbatim}
few players, but many shots per players
\end{verbatim}
  \item
    debiasing agrees and players exhibit streakiness
  \end{enumerate}
\item
\begin{verbatim}
belief in the hot hand
\end{verbatim}
  \begin{enumerate}
  \def\labelenumii{\alph{enumii}.}
  \item
\begin{verbatim}
GVT: players believe in hot hand, hot hand does not exist
\end{verbatim}
  \item
\begin{verbatim}
this: of course people can still overreact to a few consecutive makes, but it is not completely a fallacy
\end{verbatim}
  \item
\begin{verbatim}
"an understanding of the extent to which decision makers' beliefs and behavior do not corecposnd to the actual degree of hot hand shooting may have important implications for decision making more generally.
\end{verbatim}
  \item
\begin{verbatim}
hot hand is not a binary issue. it can exist, and belief can be too strong.
\end{verbatim}
    \begin{enumerate}
    \def\labelenumiii{\roman{enumiii}.}
    \tightlist
    \item
      ask fans, players, and decision makers. ironically, after one
      make, ppl tend 2 predict miss more, cause theyre biased towards
      alternating seqs.
    \end{enumerate}
  \item
    prediction questions are more informative of HH belief than
    qualitative ones
  \item
\begin{verbatim}
GVT calculated avg correlation of .04 from prediction/betting game. However, even with an assumed knowledge of the change in shooting states, the expected correlation would be .07, assuming pc = 0.45 and ph = 0.55.
\end{verbatim}
  \item
\begin{verbatim}
reanalysis of betting data, using pooled data instead of indiv.
\end{verbatim}
    \begin{enumerate}
    \def\labelenumiii{\roman{enumiii}.}
    \tightlist
    \item
      avg correlation bumps up to 0.07, and p\textless{}0.001
    \item
      weakness in GVT prediction exercise: under the null, the
      predictions should be more accurate than chance following streaks,
      but under the alt, if the players are correctly perceiving when
      hot state activates, then there won't be much of a difference in
      their accuracy between hot and cold states. hmmmm interesting
      logic.
    \end{enumerate}
  \item
\begin{verbatim}
2014 paper found that semi-pro players' rankings of teammates respective increases in FG% following 3-streaks are highly correlated with actual increase in performance (-0.60)
\end{verbatim}
  \item
\begin{verbatim}
players may be able to perceive hot hand in real time, which would require seeing more factors than binary streaks. (technique, body lang, etc.) "this suggest the possibility of conducting experiements in which experienced players (or coaches) are incenivixed to predict the shot outcomes of players that they are familiar with, but only predict when they feel sufficiently confident about their ability to do so accurately"
\end{verbatim}
  \end{enumerate}
\end{enumerate}
\begin{enumerate}
\def\labelenumi{\Alph{enumi}.}
\setcounter{enumi}{3}
\item
\begin{verbatim}
Gambler Application
\end{verbatim}
  \begin{enumerate}
  \def\labelenumii{\arabic{enumii}.}
  \tightlist
  \item
    gamblers mistakenly apply large sample properties to small samples
    (they expect things to even out)
  \item
    humans typically remember things in (relatively) small sample sizes,
    so sample size neglect occurs naturally (we overfit to what most
    recently happened) even for experienced decision makers.
  \item
    there is scientific reasoning to gamb fall bias tho:
  \end{enumerate}
  \begin{enumerate}
  \def\labelenumii{\alph{enumii}.}
  \item
\begin{verbatim}
when n > 4, 1110 is mathematically more likely to occur than 1111, which may explain why people believe that probability of 0 is greater than 0.5 after three 1s in a row.
\end{verbatim}
  \end{enumerate}
\item
\begin{verbatim}
Conclusion
\end{verbatim}
  \begin{enumerate}
  \def\labelenumii{\arabic{enumii}.}
  \tightlist
  \item
    gambler sees reversal in an iid process,
  \item
    while researchers see iid in a momentous process
  \end{enumerate}
\end{enumerate}
Jagacinski, Newell, \& Isaac (1979)
\begin{enumerate}
\def\labelenumi{\Alph{enumi}.}
\tightlist
\item
  Abstract
\end{enumerate}
\begin{enumerate}
\def\labelenumi{\arabic{enumi}.}
\item
\begin{verbatim}
college ballers predicting their own shots outcome either before release, right after release, or halfway to the basket, were no better at predicting than passive observers.
\end{verbatim}
\item
\begin{verbatim}
some evidence found for seq. dep., but not to the degree one found in bball lore
\end{verbatim}
\end{enumerate}
\begin{enumerate}
\def\labelenumi{\Alph{enumi}.}
\setcounter{enumi}{1}
\tightlist
\item
  Intro
\end{enumerate}
\begin{enumerate}
\def\labelenumi{\arabic{enumi}.}
\item
\begin{verbatim}
1977 paper found bball players cannot predict immediate performance outcomes better than an observer.
\end{verbatim}
\item
\begin{verbatim}
but they can better predict their own outcomes during the execution of their action.
\end{verbatim}
  \begin{enumerate}
  \def\labelenumii{\alph{enumii}.}
  \tightlist
  \item
    internal, non-visual cues from their own receptive feedback
  \end{enumerate}
\item
\begin{verbatim}
signal detection analysis (sensitivity = true positive rate)
\end{verbatim}
  \begin{enumerate}
  \def\labelenumii{\alph{enumii}.}
  \tightlist
  \item
    two motivations: player sensitivity and existence of seq. dep.
  \end{enumerate}
\end{enumerate}
\begin{enumerate}
\def\labelenumi{\Alph{enumi}.}
\setcounter{enumi}{2}
\tightlist
\item
  the experiment:
\end{enumerate}
\begin{enumerate}
\def\labelenumi{\arabic{enumi}.}
\item
\begin{verbatim}
subjects: 3 pairs of U of Ill. grad students who played mens college ball as undergrads
\end{verbatim}
\item
\begin{verbatim}
apparatus:
\end{verbatim}
  \begin{enumerate}
  \def\labelenumii{\alph{enumii}.}
  \tightlist
  \item
    portable wooden panel with sensors that shut off lights when a
    basketball passed thru.
  \item
    placed at midpoint of shot location, in a regular bball gym
  \end{enumerate}
\item
\begin{verbatim}
procedure:
\end{verbatim}
  \begin{enumerate}
  \def\labelenumii{\alph{enumii}.}
  \tightlist
  \item
    10 sessions, approx 1.5 hr each, one pair per session.
  \item
    subjects alternate roles of shooter and observer for 6 blocks of 60
    trials.
  \item
    session 1 is for practice, while 2-10 were experimental
  \item
    in prax, there was lights on, lights off on midpt, and lights off on
    release. to familiarize subjects with conditions, and find distance
    with 55\% success rate
  \item
    both subjects would predict outcome on 6-point scale.
  \item
    both subjects had headphones with white noise playing
  \item
    shots were taken every 10ish seconds.
  \item
    incentive of 5 cents per successful shot, and -10 for each late
    prediction
  \item
    540*3 = 1620 shots per person?
  \end{enumerate}
\item
\begin{verbatim}
results
\end{verbatim}
  \begin{enumerate}
  \def\labelenumii{\alph{enumii}.}
  \item
    signal detection analysis
  \item
\begin{verbatim}
the measure of sensitivity (A_g) was area under ROC curve
\end{verbatim}
  \end{enumerate}
  \begin{enumerate}
  \def\labelenumii{\roman{enumii}.}
  \setcounter{enumii}{1}
  \tightlist
  \item
    average sensitivities were 0.5ish (approx chance) for pre-release
    pred.
  \item
    but good for at release or post release
  \item
    shooters were not more sensitive than observers
  \end{enumerate}
  \begin{enumerate}
  \def\labelenumii{\alph{enumii}.}
  \setcounter{enumii}{1}
  \item
    seq dep analysis
  \item
\begin{verbatim}
4 subjects had a significant p-value for alt hyp: p(H|H) > p(H|M)
\end{verbatim}
  \end{enumerate}
  \begin{enumerate}
  \def\labelenumii{\roman{enumii}.}
  \setcounter{enumii}{1}
  \tightlist
  \item
    in general, subjects as shoots \& obs were optimistic in that
    p(Hpred\textbar{}H) \textgreater{} p(Hpred\textbar{}M)
  \end{enumerate}
\item
\begin{verbatim}
discussion
\end{verbatim}
  \begin{enumerate}
  \def\labelenumii{\alph{enumii}.}
  \tightlist
  \item
    equivalent sensitivity in shot prediction from shooters and
    observers suggests that shooters can't use internal cues to predict
    shot EVEN WHEN THE LIGHTS WERE OFF AFTER RELEASE!!!
  \item
    only subjects 5 \& 6 were slightly above chance in predictions as
    shooters when lights stayed on
  \item
    OFF 1st \textless{} OFF 2nd \textless{} OFF 3rd
  \item
    weak positive recency (streakiness, for makes or misses) WAS
    FOUND!!!! occurred mostly in OFF trial.
  \item
    optimism thing occurred mostly in the ON trial, even though it would
    have been more justified in the OFF trial. suggests that shooters'
    beliefs in hotness may depend on other factors such as bball lore
    rather than visual feedback.
  \item
    possible that the lore is a result of lack of awareness of
    statistical fluctuations within a game, players usually take a small
    number of shots, so it would take very large changes in hot hit prob
    to be measurable in-game. this controlled data does not suggest
    these large changes exist.
  \end{enumerate}
\end{enumerate}
Albert (1993)
\begin{enumerate}
\def\labelenumi{\Alph{enumi}.}
\tightlist
\item
  intro
\end{enumerate}
\begin{enumerate}
\def\labelenumi{\arabic{enumi}.}
\item
\begin{verbatim}
Albright used 2 models on 501 players: 
\end{verbatim}
  \begin{enumerate}
  \def\labelenumii{\alph{enumii}.}
  \tightlist
  \item
    ``null'' binom model with test stats related to streaks
  \item
    logis regress using predictors abt ``recent success'' aka history,
    \& situation
  \item
    concluded that few players were signif streaky; not enough to reject
    null tho
  \item
    streakiness effects are small in magnitude if they exist (overstated
    by media)
  \item
    concerns about power (signal detection)
  \end{enumerate}
\item
\begin{verbatim}
simple technique: look for peaks and valleys in moving avg plot
\end{verbatim}
\end{enumerate}
\begin{enumerate}
\def\labelenumi{\Alph{enumi}.}
\setcounter{enumi}{1}
\tightlist
\item
  Situational Variables
\end{enumerate}
\begin{enumerate}
\def\labelenumi{\arabic{enumi}.}
\item
\begin{verbatim}
control for factors such as Home/Away, pitcher handedness, runners on bases, etc.
\end{verbatim}
\item
\begin{verbatim}
forward selection to determine significant ones for each player!
\end{verbatim}
\item
\begin{verbatim}
predictor measuring recent success (hits in last 20 AB) the effect was negative! :O
\end{verbatim}
\end{enumerate}
\begin{enumerate}
\def\labelenumi{\Alph{enumi}.}
\setcounter{enumi}{2}
\tightlist
\item
  does hit prob change across season?
\end{enumerate}
\begin{enumerate}
\def\labelenumi{\arabic{enumi}.}
\item
\begin{verbatim}
pitcher ERA seemed to be the only generally signif predictor. try diff approach.
\end{verbatim}
\item
\begin{verbatim}
if we ignore all situational & historical vars, then we get binom trial
\end{verbatim}
\item
\begin{verbatim}
moving average using d-length bins, also lag-d autocorr of moving avgs (d=4)
\end{verbatim}
\end{enumerate}
\begin{enumerate}
\def\labelenumi{\Alph{enumi}.}
\setcounter{enumi}{3}
\tightlist
\item
  one model for streakiness
\end{enumerate}
\begin{enumerate}
\def\labelenumi{\arabic{enumi}.}
\item
\begin{verbatim}
Markov sqitching again!
\end{verbatim}
\item
\begin{verbatim}
latent variables Z1.ZT. Zi = 1(hot during game i)
\end{verbatim}
\item
\begin{verbatim}
symmetric transition matrix with rows ((0.9, 0.1),(0.1,0.9)) (pswitch = 0.1)
\end{verbatim}
\item
\begin{verbatim}
after they run on real data, they sim binom data to observe false negative possibility
\end{verbatim}
\item
\begin{verbatim}
"random data can display similar patterns"
\end{verbatim}
\end{enumerate}
\begin{enumerate}
\def\labelenumi{\Alph{enumi}.}
\setcounter{enumi}{4}
\tightlist
\item
  conclusion
\end{enumerate}
\begin{enumerate}
\def\labelenumi{\arabic{enumi}.}
\item
\begin{verbatim}
difficult to distinguish batters from coin tosses ???
\end{verbatim}
\item
\begin{verbatim}
doesn't mean streakiness doesn't exist, just that it's subtle.
\end{verbatim}
\item
\begin{verbatim}
model could improve by having more than two states
\end{verbatim}
\end{enumerate}
Albert (2013)
\begin{enumerate}
\def\labelenumi{\Alph{enumi}.}
\tightlist
\item
  Intro
\end{enumerate}
\begin{enumerate}
\def\labelenumi{\arabic{enumi}.}
\item
\begin{verbatim}
sneaky patterns in baseball
\end{verbatim}
  \begin{enumerate}
  \def\labelenumii{\alph{enumii}.}
  \tightlist
  \item
    many ways of measuring success (hits, strikeouts, HR)\\
  \item
    many ways of measuring streakiness, (runs, variation of movavgs, BF)
    \& they all may vary in how well they can detect deviation from a
    binomial model
  \item
    under null mod, some players will appear streaky by chance
    (multiplicity). so look at streakiness dist. btwn all players to see
    if there is really streakiness
  \end{enumerate}
\item
\begin{verbatim}
overview of the paper (Albert, 2008)
\end{verbatim}
  \begin{enumerate}
  \def\labelenumii{\alph{enumii}.}
  \tightlist
  \item
    binary outcomes grouped into bins. and testing for diff in probs
    across bins.
  \item
    measure lack of randomness (clumpiness) by sum2 of spacings, then
    perm test
  \item
    natural model for spacings assumes iid geom dists with diff probs.
  \item
    streaky mod assumes probs are diff and dist according to beta dist.
  \end{enumerate}
\end{enumerate}
\begin{enumerate}
\def\labelenumi{\Alph{enumi}.}
\setcounter{enumi}{1}
\tightlist
\item
  Binary seq and ``spacings''
\end{enumerate}
\begin{enumerate}
\def\labelenumi{\arabic{enumi}.}
\item
\begin{verbatim}
intro
\end{verbatim}
  \begin{enumerate}
  \def\labelenumii{\alph{enumii}.}
  \tightlist
  \item
    consider binary data for a player. spacings \{y\}1n = num(0)s btwn
    conseq 1s 101000001100 ??? y=\{0,1,5,0,2\}
  \end{enumerate}
\item
\begin{verbatim}
classical perm test
\end{verbatim}
  \begin{enumerate}
  \def\labelenumii{\alph{enumii}.}
  \item
    to quantify the amount of non-random streakiness in data
  \item
    clump stats u can use: runs, bins, lag-1 autocorr, fxn of spacings
    f(y) like (SSq(y), Entropy(y), sum of 3 largest spacings, etc.
  \item
    here they use SSqs. S = SUM(yi2)
  \item
    given n1 successes, there are (n choose n1) arrangements
  \item
    for each combination simulated:
  \item
\begin{verbatim}
randomly permute 0s and 1s and compute clumpiness stat S
\end{verbatim}
  \end{enumerate}
  \begin{enumerate}
  \def\labelenumii{\roman{enumii}.}
  \setcounter{enumii}{1}
  \tightlist
  \item
    repeat m times (they used m=1000) to get approx null distribution of
    S
  \item
    then calculate p-value of Sobs\\
  \end{enumerate}
\item
\begin{verbatim}
Bayesian "test" (BF)
\end{verbatim}
  \begin{enumerate}
  \def\labelenumii{\alph{enumii}.}
  \tightlist
  \item
    BFstreak = p(y\textbar{}Mstreak)/p(y\textbar{}Mconst) = amount of
    support for Mstreak over Mconst
  \end{enumerate}
\end{enumerate}
\begin{enumerate}
\def\labelenumi{\Alph{enumi}.}
\setcounter{enumi}{2}
\tightlist
\item
  Actually doing a Bayesian test
\end{enumerate}
\begin{enumerate}
\def\labelenumi{\arabic{enumi}.}
\item
\begin{verbatim}
Consistent and streaky models
\end{verbatim}
  \begin{enumerate}
  \def\labelenumii{\alph{enumii}.}
  \item
    since yI represents discrete waiting time (in units of ``outs'') we
    use geom
  \item
\begin{verbatim}
f(y1|pi) = pi(1-pi)^(yi) where y is any natural number
\end{verbatim}
  \item
    consistent model
  \item
\begin{verbatim}
if not streaky then all pi are equal
\end{verbatim}
  \end{enumerate}
  \begin{enumerate}
  \def\labelenumii{\roman{enumii}.}
  \setcounter{enumii}{1}
  \item
    unknown constant hitting prob p has the prior g(p) = 1/(p(1-p))
  \item
    streaky model
  \item
\begin{verbatim}
pi vary
\end{verbatim}
  \item
    pi \textasciitilde{} Beta(K\eta, K(1-\eta) for all i (K=precision,
    \eta=mean)
  \item
    as K ??? \inf, Mk ??? Mconst
  \end{enumerate}
\item
\begin{verbatim}
BF
\end{verbatim}
  \begin{enumerate}
  \def\labelenumii{\alph{enumii}.}
  \tightlist
  \item
    yeah just showing calculus methods and stuff
  \end{enumerate}
\item
\begin{verbatim}
Specifying K (difficult to assess)
\end{verbatim}
  \begin{enumerate}
  \def\labelenumii{\alph{enumii}.}
  \item
    calculate BF's for varying K's
  \item
\begin{verbatim}
K inc ??? BF inc
\end{verbatim}
  \item
    you could also do a true subjective prior
  \item
\begin{verbatim}
make guess at std dev of pi's, then calc K using empirical mean \eta.
\end{verbatim}
  \end{enumerate}
  \begin{enumerate}
  \def\labelenumii{\roman{enumii}.}
  \setcounter{enumii}{1}
  \tightlist
  \item
    n=100, \eta = approx 0.3, prior sd = 0.1 ??? prior K = 3
  \end{enumerate}
\end{enumerate}
\begin{enumerate}
\def\labelenumi{\Alph{enumi}.}
\setcounter{enumi}{3}
\tightlist
\item
  Patterns of Streakiness in Hits
\end{enumerate}
\begin{enumerate}
\def\labelenumi{\arabic{enumi}.}
\item
\begin{verbatim}
streakiness in 2011 data
\end{verbatim}
  \begin{enumerate}
  \def\labelenumii{\alph{enumii}.}
  \tightlist
  \item
    more at-bats ??? lower BF ??? more evidence towards null
  \item
    p-vals and BF's generally agree with each other.
  \end{enumerate}
\item
\begin{verbatim}
alternative BF method (by Albert (2008))
\end{verbatim}
  \begin{enumerate}
  \def\labelenumii{\alph{enumii}.}
  \tightlist
  \item
    group data into bins (nb=10) and see if probs change much across
    bins
  \item
    used same beta dist for p of each bin. results were similar ????
  \end{enumerate}
\item
\begin{verbatim}
comparing to chance
\end{verbatim}
  \begin{enumerate}
  \def\labelenumii{\alph{enumii}.}
  \tightlist
  \item
    calced BFs on 200 simulations of 2011 season with all batters using
    null model
  \item
    expect ??-error rate of 19ish\% under null model.this is LARGER than
    theobserved proportion of streaky hitters (16\%).???
  \item
    ``predictive p-value'' = prob that simmed ??-errors exceed observed
    positives
  \end{enumerate}
\end{enumerate}
\begin{enumerate}
\def\labelenumi{\Alph{enumi}.}
\setcounter{enumi}{4}
\tightlist
\item
  changing definition of success: strikeouts instead of hits
\end{enumerate}
\begin{enumerate}
\def\labelenumi{\arabic{enumi}.}
\item
\begin{verbatim}
patterns in 2011 season
\end{verbatim}
  \begin{enumerate}
  \def\labelenumii{\alph{enumii}.}
  \tightlist
  \item
    hitting is not indicative of skill, while striking out indicates a
    lack of skill
  \item
    expect ??-error rate of 17ish\% under null model.this is SMALLER
    than the observed proportion of streaky hitters (19.5\%).????
  \end{enumerate}
\item
\begin{verbatim}
looking deeper into strikeout data
\end{verbatim}
  \begin{enumerate}
  \def\labelenumii{\alph{enumii}.}
  \tightlist
  \item
    observing indiv players. instead of just fraction of streaky players
    in a season.
  \item
    only two players were streaky for \textgreater{} 2 seasons.only 12
    for 2 consec seasons
  \end{enumerate}
\end{enumerate}
\begin{enumerate}
\def\labelenumi{\Alph{enumi}.}
\setcounter{enumi}{5}
\tightlist
\item
  conclusion
\end{enumerate}
\begin{enumerate}
\def\labelenumi{\arabic{enumi}.}
\item
\begin{verbatim}
a new way of presenting streakiness.spacings
\end{verbatim}
\item
\begin{verbatim}
looked at moving percentages, not correlations
\end{verbatim}
\item
\begin{verbatim}
BF quantifies evidence for null or alt.freq test only quantifies evidence 
\end{verbatim}
\end{enumerate}
Mike West \& Harrison (1997)

M. West, Harrison, \& Migon (1985)

Prado \& West (2010)

\chapter{Data}\label{data}

The data for this analysis is from SportVU, a player-tracking system
from STATS, LLC. that provides precise coordinates for all ten players
and the ball 25 times per second. The Duke University Men's Basketball
team permitted the use of their SportVU data from the \emph{blank} to
\emph{blank} basketball seasons. However, since the ability to record
this data depends on six specialized tracking cameras, Duke does not
have this data for every game they play---only home games, and a few
road games. Therefore, there is some missing data \emph{blank}.

Each game has the following associated files:
\begin{itemize}
\item
  Box Score Optical:
\item
  Final Sequence Optical (one for each half and each overtime period):
\item
  Final Sequence Play-by-Play Optical:
\end{itemize}
\chapter{Procedure}\label{proc}

Using data that consists of a time-stamped sequence of shot locations
and outcomes, we develop a novel analysis using dynamic generalized
linear models, or DGLMs (West, Migon, and Harrison, 1985) (West and
Harrison, 1997), for binary time series. These models allow for
time-varying shot attempt and success parameters within a game and
between games. The models provide flexibility in adapting to changes in
the frequency of shot attempts, and to changes in scoring probabilities
conditional on a shot attempt. The formal Bayesian analysis allowe us to
produce full quantified inferences on such patterns over time, with
probabilistic summaries of the within-game and between-game outcomes.
For each game, this will be a full statistical characterization and
quantification of the player shooting tendencies and outcomes, which
provides understanding of inherent variability (or ``randomness'') for
the player, and formal assessments of differences in patterns
game-to-game.

I am also considering a two-state Hidden Markov Switching Model that
parallels and extends these models; the two potential states for each
shot attempt are a ``high'' probability or a ``low'' probability of
making the next shot, given the features of the current and previous
possessions.

\chapter{Models}\label{model}

Using data that consists of a time-stamped sequence of shot locations
and outcomes, we develop a novel analysis using dynamic generalized
linear models, or DGLMs (West, Migon, and Harrison, 1985) (West and
Harrison, 1997), for binary time series. These models allow for
time-varying shot attempt and success parameters within a game and
between games. The models provide flexibility in adapting to changes in
the frequency of shot attempts, and to changes in scoring probabilities
conditional on a shot attempt. The formal Bayesian analysis allowe us to
produce full quantified inferences on such patterns over time, with
probabilistic summaries of the within-game and between-game outcomes.
For each game, this will be a full statistical characterization and
quantification of the player shooting tendencies and outcomes, which
provides understanding of inherent variability (or ``randomness'') for
the player, and formal assessments of differences in patterns
game-to-game.

I am also considering a two-state Hidden Markov Switching Model that
parallels and extends these models; the two potential states for each
shot attempt are a ``high'' probability or a ``low'' probability of
making the next shot, given the features of the current and previous
possessions.

\chapter{Results}\label{results}

Using data that consists of a time-stamped sequence of shot locations
and outcomes, we develop a novel analysis using dynamic generalized
linear models, or DGLMs (West, Migon, and Harrison, 1985) (West and
Harrison, 1997), for binary time series. These models allow for
time-varying shot attempt and success parameters within a game and
between games. The models provide flexibility in adapting to changes in
the frequency of shot attempts, and to changes in scoring probabilities
conditional on a shot attempt. The formal Bayesian analysis allowe us to
produce full quantified inferences on such patterns over time, with
probabilistic summaries of the within-game and between-game outcomes.
For each game, this will be a full statistical characterization and
quantification of the player shooting tendencies and outcomes, which
provides understanding of inherent variability (or ``randomness'') for
the player, and formal assessments of differences in patterns
game-to-game.

I am also considering a two-state Hidden Markov Switching Model that
parallels and extends these models; the two potential states for each
shot attempt are a ``high'' probability or a ``low'' probability of
making the next shot, given the features of the current and previous
possessions.

\chapter{Discussion}\label{disc}

Using data that consists of a time-stamped sequence of shot locations
and outcomes, we develop a novel analysis using dynamic generalized
linear models, or DGLMs (West, Migon, and Harrison, 1985) (West and
Harrison, 1997), for binary time series. These models allow for
time-varying shot attempt and success parameters within a game and
between games. The models provide flexibility in adapting to changes in
the frequency of shot attempts, and to changes in scoring probabilities
conditional on a shot attempt. The formal Bayesian analysis allowe us to
produce full quantified inferences on such patterns over time, with
probabilistic summaries of the within-game and between-game outcomes.
For each game, this will be a full statistical characterization and
quantification of the player shooting tendencies and outcomes, which
provides understanding of inherent variability (or ``randomness'') for
the player, and formal assessments of differences in patterns
game-to-game.

I am also considering a two-state Hidden Markov Switching Model that
parallels and extends these models; the two potential states for each
shot attempt are a ``high'' probability or a ``low'' probability of
making the next shot, given the features of the current and previous
possessions.

\backmatter

\chapter*{References}\label{references}
\addcontentsline{toc}{chapter}{References}

\markboth{References}{References}

\noindent

\setlength{\parindent}{-0.20in} \setlength{\leftskip}{0.20in}
\setlength{\parskip}{8pt}

\hypertarget{refs}{}
\hypertarget{ref-albert93}{}
Albert, J. (1993). Statistical analysis of hitting streaks in baseball:
Comment. \emph{Journal of the American Statistical Association},
\emph{88}(424), 1184--1188.

\hypertarget{ref-albert13}{}
Albert, J. (2013). Looking at spacings to assess streakiness.
\emph{Journal of Quantitative Analysis in Sports}, \emph{9}(2), 1--13.

\hypertarget{ref-albert99}{}
Albert, J., \& Williamson, P. (1999). Using model/data simulations to
detect streakiness. \emph{The American Statistician}, \emph{55}, 41--50.

\hypertarget{ref-bareli06}{}
Bar-Eli, M., Avugos, S., \& Raab, M. (2006). Twenty years of ``hot
hand'' research: Review and critique. \emph{Psychology of Sport and
Exercise}, \emph{7}, 525--553.

\hypertarget{ref-gilovich85}{}
Gilovich, T., Vallone, R., \& Tversky, A. (1985). The hot hand in
basketball: On the misperception of random sequences. \emph{Cognitive
Psychology}, \emph{17}, 295--314.

\hypertarget{ref-jagacinski79}{}
Jagacinski, R. J., Newell, K. M., \& Isaac, P. D. (1979). Predicting the
success of a basketball shot at various stages of execution.
\emph{Journal of Sport Psychology}, \emph{1}, 301--310.

\hypertarget{ref-koehler03}{}
Koehler, J. J., \& Conley, C. A. (2003). The ``hot hand'' myth in
professional basketball. \emph{Journal of Sport and Exercise
Psychology}, \emph{25}, 253--259.

\hypertarget{ref-miller16}{}
Miller, J. B., \& Sanjurjo, A. (2016). Surprised by the gambler's and
hot hand fallacies? A truth in the law of small numbers. \emph{IGIER
Working Paper No. 552}.

\hypertarget{ref-west10}{}
Prado, R., \& West, M. (2010). \emph{Time series: Modelling, computation
\& inference}. Chapman \& Hall/CRC Press.

\hypertarget{ref-wetzels16}{}
Ryan Wetzels, e. a. (2016). A bayesian test for the hot hand phenomenon.
\emph{Journal of Mathematical Psychology}, \emph{72}, 200--209.

\hypertarget{ref-west97}{}
West, M., \& Harrison, P. J. (1997). \emph{Bayesian forecasting \&
dynamic models} (2nd ed.). Springer Verlag.

\hypertarget{ref-west85}{}
West, M., Harrison, P. J., \& Migon, H. S. (1985). Dynamic generalised
linear models and bayesian forecasting (with discussion). \emph{Journal
of the American Statistical Association}, \emph{80}, 73--97.


% Index?

\end{document}
